%SEITENLAYOUT, SCHRIFT:
%\RequirePackage{fix-cm}							%Schrift wirkt weniger blass
\documentclass[
	a4paper,
	%12pt,														%andere Schriftgr��e
	%DIV11,														%Devisor des Satzspiegels
	%parskip, 												%Deutsche Absatzformatierung
	%twoside,
	%openany,													%oder openright (Voreinstellung bei scrbook)
	%headsepline,											%Trennline zum Seitenkopf
	captions=tableheading							%Tabellenbeschriftungen �ber der Tabelle
	%bibtotoc, 												%Literaturverzeichnis im Inhalt
	%draft														%Dokument als Entwurf erstellen
	]{scrartcl}												%KOMA-Script Klassen: scrbook (zweiseitig), scrreprt, scrartcl (keine Titelseite), scrlettr
%\pagestyle{headings}								%Kopfzeilen
%\usepackage{geometry}							%Seitenr�nder ver�ndern mit \geometry:
%\geometry{a4paper, left=2cm, right=2cm, top=2cm, bottom=2cm} 

%SPRACHE, SCHRIFTART:
%\usepackage[T1]{fontenc}						%erweiterter Buchstabensatz mit Umlauten
%\usepackage[latin1]{inputenc}			%Eingabe �ber deutsche Tastatur (Umlaute)
%\usepackage{lmodern}								%neue Schriftfamilie "Latin Modern"
%\usepackage[ngerman]{babel}				%babel: �berschriften (Inhaltsverzeichnis) auf Deutsch, ngerman: neue Rechtschreibung

%FORMELN, TEXT:
%\usepackage{paralist}							%Aufz�hlungen editieren
\usepackage{amsmath, amssymb}				%Mathe Formelsatz, amssymb: mathematische Sonderzeichen
%\usepackage{upgreek}								%aufrechte griechische Buchstaben mit $\up...$
\usepackage{units}									%Einheiten richtig darstellen und \unitfrac[wert]{z�hler}{nenner}
%\usepackage{eurosym}								%offizielles Euro Symbol mit \EUR{}
%\usepackage{mathptmx}							%Schriftart f�r Formeln

%GRAFIKEN, TABELLEN:
\usepackage{graphicx}								%Grafiken/Bilder einbetten
\usepackage{booktabs}								%sch�nere Tabellen
\usepackage[bf]{caption}						%Bildunterschriften und Tabellen�berschriften anpassen
\renewcommand{\captionfont}{\small}	%kleinere Bildunterschriften
\usepackage{subfigure}
\usepackage{placeins}								%mit \FloatBarrier Gleitobjekte begrenzen
%\usepackage{threeparttable}				%Fussnoten in Tabellen
%\usepackage{wrapfig}
%\usepackage{longtable}							%lange Tabellen automatisch umbrechen
%\usepackage{float}									%mit [H] Bilder zwingend an dieser Stelle einbinden

%PDF-OPTIONEN:
%\usepackage{pdfpages}							%pdf-Dateien einbinden mit \includepdf[pages=1-4,landscape]{datei.pdf}
%\usepackage{hyperref}							%Verlinkung im .pdf Dokument
%\usepackage[all]{hypcap}						%Bilder, nicht Bildunterschriften verlinken
%\hypersetup{
	%pdftitle={},
	%pdfauthor={},
	%pdfborder=0 0 0, 								%keine Box um die Links
	%colorlinks=true,									%verlinkter Text farbig
	%pdfstartpage={1}, 								%Startseite des Dokuments
	%plainpages=false,	 							%empfohlen bei r�mischer Seitennummerierung zu Beginn
	%bookmarksnumbered=true						%Kapitelnummerierung in "Lesezeichen" Liste
	%}